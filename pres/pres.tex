\documentclass[handout]{beamer}


\usetheme{Warsaw}
\beamertemplatenavigationsymbolsempty
\AtBeginSubsection[]
{
 \begin{frame}{Table of Contents}
  \tableofcontents[currentsection,currentsubsection]
 \end{frame}
}


\usepackage[utf8]{inputenc}
\usepackage{listings}
\usepackage{amsmath}


\lstset{basicstyle=\footnotesize}


\title{Introduction à la programmation fonctionnelle}
\subtitle{map n00b2pgm audience}
\institute{Nantes Functional Programming Group}
\date{14 mai 2012}


\begin{document}
 \frame{\titlepage}





 \section{La pensée fonctionnelle}
 \subsection{Le clash des visions}
  \begin{frame}
   La programmation fonctionnelle quand on vient de l'impératif, c'est avant
   tout un autre paradigme.
   \begin{block}{paradigme @Wikipédia}
   Un paradigme est une représentation du monde, une manière de voir les choses.
   \end{block}
  \end{frame}


  \begin{frame}{Le paradigme impératif}
   On dicte à l'ordinateur des modifications à exécuter sur l'état du
   programme :
   \pause
   \begin{itemize}[<+->]
    \item on déclare des variables (allocation de mémoire)
    \item on les transforme (modification de la mémoire allouée)
    \item on les détruit (désallocation de mémoire)
   \end{itemize}
  \end{frame}


  \begin{frame}{Le paradigme fonctionnel}
   On abstrait une partie de l'implémentation pour se concentrer sur les
   données : on déclare la relation qui existe entre les données que l'on reçoit
   et celles que l'on recherche
  \end{frame}


  \begin{frame}[fragile]{Intuition de la différence entre les deux}
   Au lieu de dire comment on fait les choses, on dit simplement que les choses
   doivent être faites. Le reste relève du talent des codeurs de GHC (ou
   équivalent pour les autres langages :)).
   
   \pause
   
   Par exemple, pour calculer la somme des nombres allant de 1 à n, en C à
   gauche et en Haskell à droite, on dirait :
   \begin{columns}[t] % the "c" option specifies center vertical alignment
    \column{.5\textwidth} % column designated by a command
     \begin{lstlisting}[language=C]
int f(int n) {
  int result = 0, i = 1;
  for(; i <= n; i++)
    result += i;
  return result;
}
     \end{lstlisting}
    \column{.5\textwidth}
     \begin{lstlisting}[language=Haskell]
f :: Integer -> Integer
f n = sum [1..n]
     \end{lstlisting}
   \end{columns}
\end{frame}





 \section{Les concepts clefs}
 \subsection{Les fonctions}
  \begin{frame}{Dégrossir le concept de fonction}
   Une fonction est une boîte noire qui transforme un objet en un autre
   objet.
  \end{frame}


  \begin{frame}{Un peu de formalisme}
   Une fonction se caractérise par :
   \pause
   \begin{itemize}[<+->]
    \item son domaine : l'ensemble duquel est issu son argument
    \item son codomaine : l'ensemble auquel appartient le résultat
    \item la manière dont elle transforme l'argument en le résultat
   \end{itemize}
  \end{frame}


  \begin{frame}[fragile]{Les fonctions : exemple avec la fonction carré}
   \begin{block}{En français}
    \begin{itemize}
     \item son argument est un entier naturel
     \item son résultat est un entier naturel
     \item elle retourne son argument au carré
    \end{itemize}
   \end{block}
   \pause
   \begin{block}{En maths}
    \begin{tabular}{l l l l}
     \texttt{carré :} & $\mathbb{N}$ & $\rightarrow$ & $\mathbb{N}$ \\
                      & $x$          & $\mapsto$     & $x^2$ \\
    \end{tabular}
   \end{block}
   \pause
   \begin{block}{En Haskell}
    \begin{lstlisting}[language=Haskell]
square :: Integer -> Integer
square n = n ^ 2
    \end{lstlisting}
   \end{block}
\end{frame}


  \begin{frame}{En quoi sont-elles un concept clef}
   Pour décrire les données que l'on souhaite obtenir à partir des données de
   départ dans un programme, on exprime des transformations.
   
   Les fonctions sont la manière la plus simple de les spécifier.
  \end{frame}


  \begin{frame}{Ce qu'est un programme fonctionnel}
   On pourrait voir un programme fonctionnel comme un enchaînement d'appels de
   fonctions sur les données de départ :
   
   \[
    d_i \overset{f}   {\longrightarrow}
    d_1 \overset{g}   {\longrightarrow}
    d_2 \overset{h} {\longrightarrow}
    d_f
   \]
  \end{frame}


  \begin{frame}{La composition}
   Composer deux fonctions, c'est les "fusionner" pour obtenir la fonction
   qui se définit par l'application des deux fonctions d'origine.
   \pause
   \begin{block}{En version matheuse}
    \begin{tabular}{l l l l l l l l l l}
     si & $f :$ & $B$ & $\rightarrow$ & $C$ &
     et & $g :$ & $A$ & $\rightarrow$ & $B$ \\
        &       & $b$ & $\mapsto$     & $c$ &
        &       & $a$ & $\mapsto$     & $b'$ \\
    \end{tabular} \newline \newline
    \begin{tabular}{l l l l l}
     alors & $f \circ g :$ & $A$ & $\rightarrow$ & $C$ \\
           &               & $a$ & $\mapsto$     & $f(g(a))$ \\
    \end{tabular}
   \end{block}
  \end{frame}


  \begin{frame}{Ce qu'est un programme fonctionnel bis}
   On peut reprendre la vision proposée d'un programme fonctionnel :
   
   \[
    d_i \overset{f}   {\longrightarrow}
    d_1 \overset{g}   {\longrightarrow}
    d_2 \overset{h} {\longrightarrow}
    d_f
   \]
   Et en proposer une autre, cette fois en utilisant la composition de
   fonctions :
    \[
    d_i \overset{h \circ g \circ f}   {\longrightarrow} d_f
   \]
   \pause
   Modularité : si on a $g = g'$, alors, 
   le programme suivant est équivalent au programme ci-dessus :
    \[
    d_i \overset{h \circ g' \circ f}   {\longrightarrow} d_f
   \]
  \end{frame}





 \subsection{La transparence référentielle}
  \begin{frame}{Ce qu'est la transparence référentielle}
   La transparence référentielle, c'est le fait de pouvoir interchanger l'appel
   d'une fonction et son résultat indifférement.
  \end{frame}


  \begin{frame}{Pourquoi seulement en fonctionnel ?}
   \begin{itemize}
    \item nécessité qu'une fonction retourne toujours le même résultat pour un
    argument donné
    \item nécessité qu'une fonction n'ait pas d'effets de bord
   \end{itemize}
   \pause
   Ces deux points ne sont souvent pas respectés en impératif. De fait, les
   fonctions impératives sont plus des procédures.
  \end{frame}


  \begin{frame}{Ce que ça implique}
   \begin{itemize}[<+->]
    \item plus facile de raisonner sur le comportement d'une fonction
    \item optimisations à l'exécution (mémoization, cache, parallélisation)
    \item testabilité garantie
   \end{itemize}
  \end{frame}





 \subsection{L'évaluation paresseuse}
  \begin{frame}{Ce qu'est l'évaluation paresseuse}
   On évalue un bloc que lorsque l'on en a besoin (donc potentiellement jamais).
  \end{frame}


  \begin{frame}{Ce que ça implique}
   \begin{itemize}
    \item un langage paresseux peut manipuler des structures de données infinies
    \item un language paresseux est \textbf{non strict}. L'appel d'une
          fonction avec un argument non défini n'est pas forcément non défini.
   \end{itemize}
  \end{frame}





 \subsection{La récursion}
  \begin{frame}{Ce qu'est la récursion}
   La récursion est le fait pour une fonction de s'appeller soi-même
   (directement ou indirectement).
   
   La définition par récurrence est quant à elle le fait de définir la valeur
   d'un élément d'une suite en fonction de la valeur de l'élément qui le
   précède. Il suffit ensuite de connaître la valeur d'un élément pour obtenir
   la valeur de tous ses successeurs.
  \end{frame}


  \begin{frame}{Intuition sur la récursion}
   Pour savoir monter à n'importe quel barreau d'une échelle, il suffit de :
   \pause
   \begin{itemize}
    \item savoir monter sur le premier barreau
    \item savoir monter d'un barreau au barreau suivant
   \end{itemize}
   \pause
   La définition par récurrence est analogue : avec une \emph{initialisation}
   et une \emph{propriété d'hérédité}, on définit toute une suite.
  \end{frame}


  \begin{frame}{Exemples}
   \begin{block}{La suite des nombres pairs $(P_n)$}
    \[
     \begin{cases}
      P_0 = 0\\
      \forall n \geq 0, \quad P_{n+1} = P_n + 2
     \end{cases}
    \]
   \end{block}
   \pause
   \begin{block}{La suite des nombres de Fibonacci $(F_n)$}
    \[
     \begin{cases}
      F_0 = 0\\
      F_1 = 1\\
      \forall n \geq 0, \quad F_{n+2} = F_{n+1} + F_n
     \end{cases}
    \]
   \end{block}
  \end{frame}


  \begin{frame}{Exemples - suite}
   On peut s'amuser avec des exemples un petit peu plus tordus -- ici on définit
   deux suites en même temps :
   \begin{block}{Les suites des nombres pairs $(P_n)$ et impairs $(I_n)$}
    \[
     \begin{array}{c c}
      \begin{cases}
       P_0 = 0\\
       \forall n \geq 0, \quad P_{n+1} = I_n + 1
      \end{cases}
      &
      \forall n \geq 0, \quad I_n = P_n + 1
      \\
     \end{array}
    \]
   \end{block}
  \end{frame}





 \subsection{La co-récursion}
  \begin{frame}{Ce qu'est la co-récursion}
   La co-récursion, on peut dire pour simplifier que c'est la récursion sur
   le résultat (le co-domaine) plutôt que l'argument (le domaine).
   \begin{block}{Rappel sur la récursion}
    Quand on utilise une définition par récurrence, le but est de successivement
    s'appeller avec des arguments "plus petits" jusqu'à atteindre un cas de
    base.
   \end{block}
   \begin{block}{La co-récursion}
    Quand on utilise une définition par co-récurrence, le but est de construire
    le résultat en s'appellant successivement avec des arguments "plus grands".
    Il peut y avoir ou non un cas d'arrêt (souvent il n'y en a pas).
   \end{block}
  \end{frame}


  \begin{frame}[fragile]{Exemples}
   \begin{block}{Rappel de récursion}
     \begin{lstlisting}[language=Haskell]
pair :: Integer -> Integer
pair 0 = 0
pair x = pair (x - 1) + 2
     \end{lstlisting}
   \end{block}
   \pause
   \begin{block}{Co-récursion}
     \begin{lstlisting}[language=Haskell]
pairs :: [Integer]
pairs = 0 : map (+ 2) pairs
     \end{lstlisting}
   \end{block}
\end{frame}


  \begin{frame}[fragile]{On peut être les deux...}
   \begin{block}{Comme map...}
     \begin{lstlisting}[language=Haskell]
map :: (a -> b) -> [a] -> [b]
map _ []     = []
map f (x:xs) = f x : map f xs
     \end{lstlisting}
   \end{block}
   \pause
   \begin{block}{... ou zipWith}
     \begin{lstlisting}[language=Haskell]
zipWith :: (a -> b -> c) -> [a] -> [b] -> [c]
zipWith f (a:as) (b:bs) = f a b : zipWith f as bs
zipWith _ _      _      = []
     \end{lstlisting}
   \end{block}
\end{frame}


  \begin{frame}[fragile]{Utilité}
   La co-récursion est extrêmement utile pour spécifier les listes infinies
   (qu'on appelle aussi streams). Quelques exemples canoniques sont:
   
   \begin{block}{Les nombres de Fibonacci}
     \begin{lstlisting}[language=Haskell]
fibs :: [Integer]
fibs = 0 : 1 : zipWith (+) fibs (tail fibs)
     \end{lstlisting}
   \end{block}
   \pause
   \begin{block}{Les factorielles}
     \begin{lstlisting}[language=Haskell]
factorials = [Integer]
factorials = 1 : zipWith (*) [1..] factorials
     \end{lstlisting}
   \end{block}
\end{frame}
\end{document}
